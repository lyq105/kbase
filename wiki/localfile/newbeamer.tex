%\documentclass[t,c,b]{beamer}
\documentclass[t]{beamer}

%\documentclass{article}

\usepackage[UTF8]{ctex}
\hypersetup{colorlinks=true,linkcolor=red}
%Theme
%\usetheme{default}
%\usetheme{Malmoe}
%\usetheme{Boadilla}
%\usetheme[height=7mm]{Rochester}
%\usetheme{Copenhagen} 
\usetheme{Warsaw} 
%\usetheme{umbc1} 
% 背景图片
%\setbeamertemplate{background canvas}{\includegraphics
%	[width=\paperwidth,height=\paperheight]{alps.jpg}}

% items enclosed in square brackets are optional; explanation below
\title[A short proof]{A short proof of Fermat's Last Theorem}
\subtitle[Errors]{Estimation of numerical errors}
\author[Y. Li]{Yiqiang Li}

%The optional argument, UMBC (in square brackets) is the short form of the institute’s name.
\institute[NWPU]{
Department of Applied Mathematics \\
Northwestern Polytechnical University, \\
Xi'An, China 710072\\[1ex]
\texttt{liyiqiang@mail.nwpu.edu.cn}
}
\date[November 2004]{November 26, 2004}



\begin{document}

%--- the titlepage frame -------------------------%
\begin{frame}[plain]
	\titlepage
\end{frame}

%--- frame 1-------------------------%
\begin{frame}{一个例子}
	\[
	\int_{-\infty}^\infty e^{-x^2} \, dx =\sqrt{\pi}
	\]	

\end{frame}

%--- frame 2 -------------------------%
\begin{frame}{Graphics} 

	Here we include three images, one each of PDF, PNG, and JPG types. 

	\begin{center} 
		%\includegraphics[width=0.3\textwidth]{image1.pdf} 
		%\includegraphics[width=0.3\textwidth]{image2.png} 
		%\includegraphics[width=0.3\textwidth]{image3.jpg} 
		\end{center} 

	\end{frame} 

	\begin{frame}{Outline of the talk (about overlay 1)} 

		\begin{itemize} 
			\item Introduction 
				\pause 
			\item Statement of the main theorem 
				\pause 
			\item Technical lemmata 
				\pause 
			\item Proof of the main theorem 
				\pause 
			\item Conclusions 
		\end{itemize} 

	\end{frame} 

	\begin{frame}{Fermat's Last Theorem (About overlay 2)} 

		In this talk I will give a very elementary proof of the 
		theorem.  I am surprised that no one else has thought of 
		this before. 
		\medskip 

		\pause 

		Fermat's Last Theorem says that the equation 
		\[ 
		x^2 + y^2 = z^2 
		\] 
		has no solution in the set of natural numbers. 
		\medskip 

		\pause 

		This is not true.  After a lengthy calculation on the 
		department's Linux machines, I have verified that within 
		the numerical accuracy of the Pentium-4 processor, we have: 
		\[ 
		5000^2 + 12000^2 = 13000^2 
		\] 

	\end{frame} 
	%--- frame --------------------------------------------------%
	\begin{frame}[label=intro]{Introduction 链接 1}

		This slide is labeled ``intro''.

	\end{frame}

	%--- frame --------------------------------------------------%
	\begin{frame}{链接 }

		If you click \hyperlink{intro}{here}, you will jump to the slide
		labeled ``intro''.

		\bigskip

		Clicking \hyperlink{intro}{\beamerbutton{here}} will also
		take you to the ``intro'' slide.
		\beamerbutton{here} 
		\beamergotobutton{here} 
		\beamerskipbutton{here} 
		\beamerreturnbutton{here} 
	\end{frame}

	\begin{frame}{Theorems and such}
		\begin{definition}
			A triangle that has a right angle is called
			a \emph{right triangle}.
		\end{definition}
		\begin{theorem}
			In a right triangle, the square of hypotenuse equals
			the sum of squares of two other sides.
		\end{theorem}

		\begin{proof}
			We leave the proof as an exercise to our astute reader.
			We also suggest that the reader generalize the proof to
			non-Euclidean geometries.
		\end{proof}
	\end{frame}

	\begin{frame}{Splitting a slide into columns}

		The line you are reading goes all the way across the slide.
		From the left margin to the right margin.  Now we are going
		the split the slide into two columns.
		\bigskip

		\begin{columns}
			\begin{column}{0.5\textwidth}
				Here is the first column.  We put an itemized list in it.
				\begin{itemize}
					\item This is an item
					\item This is another item
					\item Yet another item
				\end{itemize}
			\end{column}

			\begin{column}{0.3\textwidth}
				Here is the second column.  We will put a picture in it.
				%    \centerline{\includegraphics[width=0.7\textwidth]{image2.png}}
				\end{column}
			\end{columns}
			\bigskip

			The line you are reading goes all the way across the slide.
			From the left margin to the right margin.

		\end{frame}

		\begin{frame}{对齐Splitting a slide into columns}

			The line you are reading goes all the way across the slide.
			From the left margin to the right margin.  Now we are going
			the split the slide into two columns.
			\bigskip
			%--------------***-----------
			\begin{columns}[T]
				\begin{column}{0.5\textwidth}
					Here is the first column.  We put an itemized list in it.
					\begin{itemize}
						\item This is an item
						\item This is another item
						\item Yet another item
					\end{itemize}
				\end{column}

				\begin{column}{0.3\textwidth}
					Here is the second column.  We will put a picture in it.
					% \centerline{\includegraphics[width=0.7\textwidth]{image2.png}}
					\end{column}
				\end{columns}
			\end{frame}

			\begin{frame}[b]{Bottom alignment} 

				This is the contents of the slide. 

			\end{frame} 
			\begin{frame}[c]{Center alignment (default)} % [c] is the default 

				This is the contents of the slide. 

			\end{frame} 



			\end{document}

